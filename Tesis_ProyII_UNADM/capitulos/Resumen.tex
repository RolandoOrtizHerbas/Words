\addcontentsline{toc}{chapter}{{}Resumen}

\chapter*{Resumen}

\vspace*{-8mm}

El uso de la inteligencia artificial, y en particular del aprendizaje automático (Machine Learning), ha transformado la evaluación de riesgos en préstamos, al considerar múltiples factores como historiales de crédito y tendencias del mercado. Estos modelos permiten predecir con 
mayor precisión la probabilidad de incumplimiento, lo que ayuda
 a las instituciones financieras a ajustar las tasas de interés y asignar recursos de manera eficiente. Además, su capacidad de adaptación a cambios en los datos y condiciones del mercado los hace flexibles. No obstante, es fundamental enfrentar retos éticos y 
de privacidad para asegurar un uso justo. \medskip

Palabras clave: inteligencia artificial, aprendizaje automático, evaluación de riesgos, préstamos, previsión de incumplimiento, adaptación, ética, privacidad.

