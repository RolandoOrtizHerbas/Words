\chapter{Introducción}\label{cap1:Introducción}

\section{Antecedentes}



El presente proyecto tiene como objeto el elaborar un modelo que nos permita determinar un 
conjunto de variables que nos permitan clasificar a un cliente como confiable para ser un 
sujeto de crédito o no. Este proceso también se llevará a cabo durante el tiempo que este 
pagando el crédito que se le otorgó por lo que es estatus del cliente puede variar en el 
tiempo. \medskip

Este estudio se esta realizando para la empresa  “Wireless And Mobile Telecommunications, S. de R.L. de C.V.” 
que  es una consultora en el área de telecomunicaciones e inteligencia artificial. \medskip

“Wireless And Mobile Telecommunications”, busca potenciar su capacidad de gestión de riesgos 
en el ámbito crediticio. Con una base de clientes en constante expansión, la empresa se 
encuentra en la encrucijada de equilibrar el acceso al crédito para sus clientes con la 
necesidad de salvaguardar su salud financiera. \medskip

El contexto actual destaca la importancia de adoptar enfoques innovadores, y es en este marco 
que surge la iniciativa de implementar un modelo de riesgo de créditos basado en 
machine learning. Este enfoque moderno permitirá evaluar de manera más precisa y 
eficiente la capacidad crediticia de sus clientes, optimizando así el proceso de toma de 
decisiones. \medskip

El modelo se centrará en el análisis de datos, utilizando algoritmos de machine learning para 
identificar patrones relevantes que influyen en la solvencia crediticia. La integración de datos 
internos mejorará la robustez del modelo, ofreciendo a la empresa una herramienta ágil y 
precisa para evaluar el riesgo asociado a cada solicitud de crédito. \medskip

La implementación de este modelo no solo fortalecerá la posición financiera de la empresa, 
sino que también mejorará la experiencia del cliente al agilizar el proceso de aprobación de 
créditos. La capacidad de ofrecer respuestas rápidas y personalizadas a las solicitudes de 
crédito no solo aumentará la satisfacción del cliente, sino que también respaldará el 
crecimiento continuo de la empresa en un entorno competitivo.
\section{Justificación}

Un estudio realizado por “Wireless And Mobile Telecommunications, S. de R.L. de C.V.” 
reveló que, de una muestra de 1000 clientes, 142 incumplieron con 
el pago de sus créditos automotrices. \medskip

Dado que cada crédito promedio asciende a 300,000 pesos M.N. y, 
asumiendo que los clientes morosos dejaron de pagar el 50% 
del crédito, la pérdida estimada para la empresa
asciende a 21,000,000 pesos M.N. (142 clientes * 150,000 pesos M.N.). \medskip

Esta situación subraya la importancia de mejorar las
herramientas de evaluación de riesgo y control de morosidad.
Implementar soluciones avanzadas, como el uso de inteligencia artificial y modelos de aprendizaje automático, 
permitiría prever con mayor precisión los casos de incumplimiento 
y, por ende, reducir las pérdidas financieras. Además, una gestión proactiva del riesgo no solo optimiza la asignación de recursos, sino que también fortalece la estabilidad económica de la empresa en un entorno altamente competitivo.

\section{Objetivos}

\subsection{Objetivo general}

Desarrollar e implementar un modelo para la evaluación dinámica de la elegibilidad crediticia 
de los clientes , desde el inicio del crédito y a lo largo del tiempo, mediante periodos 
definidos de pago, con el fin de optimizar la toma de decisiones crediticias y fortalecer 
la salud financiera de la empresa.

\subsection{Objetivos específicos}

\begin{itemize}
\item Identificar variables clave para la evaluación inicial de la capacidad crediticia de los clientes al momento de solicitar un crédito.\medskip


\item Desarrollar un modelo que clasifique a los clientes como sujetos o no sujetos de crédito al inicio de la relación crediticia.\medskip

\item Establecer periodos definidos de análisis temporal, considerando variables como historial de pagos y comportamiento crediticio.\medskip

\item Recopilar y procesar datos relevantes de los clientes durante cada periodo definido, asegurando la actualización constante del modelo.\medskip

\item Refinar el modelo  a medida que se acumulan datos adicionales, mejorando la capacidad predictiva a lo largo del tiempo.\medskip

\item Evaluar la eficacia del modelo mediante métricas de desempeño, como precisión, sensibilidad y especificidad, para garantizar su confiabilidad en la toma de decisiones crediticias.\medskip

\item Implementar el modelo en el proceso de toma de decisiones crediticias, integrándolo de manera efectiva en las operaciones cotidianas.\medskip

\item Monitorear continuamente el rendimiento del modelo y realizar ajustes según sea necesario para adaptarse a cambios en el comportamiento crediticio de los clientes y en el entorno económico.\medskip

\end{itemize}